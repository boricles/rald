\documentclass{llncs}
\usepackage{algorithm}
\usepackage{algpseudocode}
\algtext*{EndFor}
\algtext*{EndIf}
\algtext*{EndProcedure}
%\usepackage[noend]{algorithmic}

\usepackage{authblk}
\usepackage{amsmath}
\usepackage{ulem}
\usepackage{color}
\usepackage{courier}
\usepackage{dl}
\usepackage{graphicx}
\usepackage{helvet}
\usepackage{ifsym}
\usepackage{multirow}
\usepackage{textcomp}
\usepackage{theorem}
\usepackage{times}
\usepackage{url}
\usepackage{wasysym}
\usepackage{paralist}
\usepackage{listings}

\usepackage[T1]{fontenc}

%\newdef{definition}{Definition}
%\newtheorem{theorem}{Theorem}
%\newtheorem{lemma}{Lemma}
%\newtheorem{proposition}{Proposition}
%\newdef{example}{Example}
\newcommand{\todo}[1]{\textcolor{red}{[#1]}\xspace}

\newcommand{\qn}[2]{\ensuremath{#1\mbox{:}#2}\xspace}

\newcommand{\tr}[3]{<\ensuremath{#1,#2,#3}>\xspace}

\newcommand{\B}{\ensuremath{\mathcal{B}}\xspace}

\newcommand{\V}{\ensuremath{\mathcal{V}}\xspace}

\renewcommand{\L}{\ensuremath{\mathcal{L}}\xspace}

\renewcommand{\KB}{\ensuremath{g}\xspace}


\newlength{\hiindent}
\setlength{\hiindent}{0.5cm}
\newcommand{\hi}[1]{\hspace*{#1 \hiindent}}

\makeatletter
\let
\@copyrightspace
\relax
\makeatother


\title{How Redundant Is It? - An Empirical Analysis on Linked Datasets}

\author{Honghan Wu\inst{1} \and Boris Villazon-Terrazas\inst{2} \and Jeff Z. Pan\inst{1} \and Jose Manuel Gomez-Perez\inst{2}}
%
\authorrunning{Wu et al.}   % abbreviated author list (for running head)
%
%%%% list of authors for the TOC (use if author list has to be modified)
%\tocauthor{Panos Alexopoulos, Manolis Wallace}
%
\institute{Department of
Computing Science, University of Aberdeen, UK \and
iSOCO, Intelligent Software Components S.A., Spain}

\begin{document}


\maketitle

\begin{abstract}
While there are some popular vocabularies widely used across linked open data, many linked data sets do not have T-Box axioms at all. How does such fact, i.e., the usage patterns on T-Boxes, affect the linked data consumption? This might be an interesting question to be asked by linked data consumers. In this paper, we analysis one particular aspect of this question i.e., redundancy analysis, on several popular datasets and vocabularies in web of data. We start with the analysis on semantic redundancies of datasets given their T-Boxes. Then, we propose useful t-box axioms which are helpful for consumption but are absent in the dataset in question. Finally, we reveal how the linkages  of the web of data, both in concept-level and instance-level, affects data set redundancies.

\end{abstract}

\section{Introduction}
\label{sec:intro}
%\subsection{What is data redundancy with linked data?}
Data redundancy has different meanings in different contexts. In database community, data redundancy means that the same piece of data is stored in \emph{multiple places} in a database system, while in information theory it means the \emph{wasted space} used to transmit certain data. In this paper, we focus on data redundancy in Linked Data, which means the \emph{wasted space} used to represent certain meaning in either stand-alone data space or the Web of Data environment.

%\subsection{Why is it of special interest to linked data consumption?}
Depending on the sceraios, data redundancy in Linked Data might have different effects to data consumption tasks. Sometimes Linked Data redundancy might be unwelcome. For example, for storage or exchange purpose, the redundancy will cause unnecessary resource consumptions, e.g. more disk spaces or longer time to download a dump file from the Web. In other occasions the redundancy of the data can be utilised to facilitate the task on hand. For example, in Ontology Based Data Access (OBDA) tasks, A-Box redundancy can be utilised to avoid unnecessary query rewritings so that the efficiency of the system can be improved. 

Redundancies in Linked Data have effects on a wide range of applications including data publishing, query answering (in SPARQL endpoints), OBDA, and ontology reasoning. Existing work~\cite{joshi2013logical,cure2014waterfowl,fernandez2013binary} either focuses on RDF data compression or provides succinct data structure for data access. Little attention has been put on making the data redundancy explicit and available to Linked Data stakeholders. For data consumers, a good knowledge about data redundancies in interested datasets will help users either make use of them or choose the best technique to avoid them. For data publishers, such knowledge will guide them in making the right decisions like reusing the right vocabularies, linking to right datasets or sometimes not linking at all. In this paper, we focus on revealing the redundancies in Linked Data by both a qualitative analysis of systematic redundancy categorisation and an quantitative analysis on various datasets covering different domains.

The rest of the paper is organised as follows. Section~\ref{sec:gp} gives an discussion about the categorisation of redundancies in Linked Data. Section~\ref{sec:method} proposes our redundancy analysis methodology which covers two different dimensions. In section~\ref{sec:related}, we briefly introduces related work. Section~\ref{sec:result} gives the detailed analysis results on real world datasets. Finally, we conclude the work in section~\ref{sec:conclusion}.
%\subsubsection{The Good}

%\subsubsection{The Bad}

%\subsubsection{The Ugly}


\section{Graph Pattern Based Redundancy Identification}
\label{sec:gp}

\section{A-Box Redundancy}
\label{sec:t-box}

\section{T-Box Rules VS. A-Box Redundancy}
\label{sec:t-box}

\section{Redundancy deviation by vocabulary Linkage }
\label{sec:vlinkage}

\section{Graph Pattern Based Redundancy Identification}
\label{sec:gp}

\section{Related Work}
\label{sec:related}
RDF compression techniques have been proposed to eliminate data redundancies. Such as RDF serialisations techniques, i.e. HDT serialisation~\cite{fernandez2013binary}, lean graphs~\cite{iannone2005optimizing} and K2-triples~\cite{alvarez2011compressed} can be used to reduce file size. Another approach is based on logical compression, such as the rule-based RDF compression~\cite{joshi2013logical}, which can be used to substantially reduce the number of triples in an RDF document. Inspired by the HDT approach, Cur{\'e} et al. proposed WaterFowl ~\cite{cure2014waterfowl} as a succinct data structure for RDF data. The OWL sameAs network was studied by Ding et al.~\cite{ding2010sameas}. The implication of sameAs links was raised but not studied. Halpin et al.~\cite{halpin2010owl} also proposed a way to analyse identity in linked data based on sameAs links. As we mentioned, none of existing work has made the data redundancies explicit and available to stakeholders of Linked Data, which is the main focus of this paper.

\section{Conclusion}
\label{sec:conclusion}

\bibliographystyle{abbrv}
\bibliography{cold}
\end{document}
