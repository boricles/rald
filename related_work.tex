RDF compression techniques have been proposed to eliminate data redundancies. Such as RDF serialisations techniques, i.e. HDT serialisation~\cite{fernandez2013binary}, lean graphs~\cite{iannone2005optimizing} and K2-triples~\cite{alvarez2011compressed} can be used to reduce file size. Another approach is based on logical compression, such as the rule-based RDF compression~\cite{joshi2013logical}, which can be used to substantially reduce the number of triples in an RDF document. Inspired by the HDT approach, Cur{\'e} et al. proposed WaterFowl ~\cite{cure2014waterfowl} as a succinct data structure for RDF data. The OWL sameAs network was studied by Ding et al.~\cite{ding2010sameas}. The implication of sameAs links was raised but not studied. Halpin et al.~\cite{halpin2010owl} also proposed a way to analyse identity in linked data based on sameAs links. As we mentioned, none of existing work has made the data redundancies explicit and available to stakeholders of Linked Data, which is the main focus of this paper.